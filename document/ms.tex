\documentclass[12pt,preprint]{aastex}

% has to be before amssymb it seems
\usepackage{color,hyperref}
\definecolor{linkcolor}{rgb}{0,0,0.5}
\hypersetup{colorlinks=true,linkcolor=linkcolor,citecolor=linkcolor,
            filecolor=linkcolor,urlcolor=linkcolor}

\usepackage{url}
\usepackage{algorithmic,algorithm}
\usepackage{multirow}

\usepackage{listings}
\definecolor{lbcolor}{rgb}{0.9,0.9,0.9}
\lstset{language=Python,
        basicstyle=\footnotesize\ttfamily,
        showspaces=false,
        showstringspaces=false,
        tabsize=2,
        breaklines=false,
        breakatwhitespace=true,
        identifierstyle=\ttfamily,
        keywordstyle=\bfseries\color[rgb]{0.133,0.545,0.133},
        commentstyle=\color[rgb]{0.133,0.545,0.133},
        stringstyle=\color[rgb]{0.627,0.126,0.941},
    }

\usepackage{amssymb,amsmath}

\newcommand{\project}[1]{{\sffamily #1}}
\newcommand{\Python}{\project{Python}}
\newcommand{\numpy}{\project{numpy}}
\newcommand{\bart}{\project{Bart}}
\newcommand{\emcee}{\project{emcee}}
\newcommand{\kepler}{\project{Kepler}}
\newcommand{\kplr}{\project{kplr}}
\newcommand{\license}{MIT License}

\newcommand{\paper}{\emph{Article}}

\newcommand{\foreign}[1]{\emph{#1}}
\newcommand{\etal}{\foreign{et\,al.}}
\newcommand{\etc}{\foreign{etc.}}

\newcommand{\Fig}[1]{Figure~\ref{fig:#1}}
\newcommand{\fig}[1]{\Fig{#1}}
\newcommand{\figlabel}[1]{\label{fig:#1}}
\newcommand{\Tab}[1]{Table~\ref{tab:#1}}
\newcommand{\tab}[1]{\Tab{#1}}
\newcommand{\tablabel}[1]{\label{tab:#1}}
\newcommand{\Eq}[1]{Equation~(\ref{eq:#1})}
\newcommand{\eq}[1]{\Eq{#1}}
\newcommand{\eqlabel}[1]{\label{eq:#1}}
\newcommand{\Sect}[1]{Section~\ref{sect:#1}}
\newcommand{\sect}[1]{\Sect{#1}}
\newcommand{\App}[1]{Appendix~\ref{sect:#1}}
\newcommand{\app}[1]{\App{#1}}
\newcommand{\sectlabel}[1]{\label{sect:#1}}
\newcommand{\Algo}[1]{Algorithm~\ref{algo:#1}}
\newcommand{\algo}[1]{\Algo{#1}}
\newcommand{\algolabel}[1]{\label{algo:#1}}

% math symbols
\newcommand{\dd}{\ensuremath{\,\mathrm{d}}}
\newcommand{\bvec}[1]{\ensuremath{\boldsymbol{#1}}}
\newcommand{\unit}[1]{\ensuremath{\mathrm{#1}}}
\newcommand{\normal}[1]{\ensuremath{\mathcal{N}(#1)}}

\newcommand{\obs}[1]{\ensuremath{\overline{#1}}}

% document symbols
\newcommand{\population}{\ensuremath{\alpha}}
\newcommand{\planet}{\ensuremath{w}}
\newcommand{\planetobs}{\ensuremath{\obs{w}}}
\newcommand{\stellar}{\ensuremath{s}}
\newcommand{\stellarobs}{\ensuremath{\obs{s}}}
\newcommand{\isobs}{\ensuremath{q}}
\newcommand{\selection}{\ensuremath{\Delta}}

% parameters
\newcommand{\period}{\ensuremath{P}}
\newcommand{\relincl}{{\ensuremath{\delta i}}}
\newcommand{\radius}{\ensuremath{r}}
\newcommand{\periodobs}{\ensuremath{\obs{P}}}
\newcommand{\rorobs}{{\ensuremath{\obs{r/\sradius}}}}
\newcommand{\impactobs}{\ensuremath{\obs{b}}}

\newcommand{\smass}{\ensuremath{M}}
\newcommand{\sradius}{\ensuremath{R}}
\newcommand{\snoise}{\ensuremath{\sigma}}
\newcommand{\incl}{\ensuremath{i}}
\newcommand{\snoiseobs}{\ensuremath{\obs{\sigma}}}
\newcommand{\sloggobs}{{\ensuremath{\obs{\log g}}}}


\begin{document}

\title{%
The \kplr\ Catalog:
Systematic probabilistic parameter estimation for every \kepler\ planet candidate
}

\newcommand{\nyu}{2}
\newcommand{\mpia}{3}
\author{%
    Daniel~Foreman-Mackey\altaffilmark{1,\nyu},
    David~W.~Hogg\altaffilmark{\nyu,\mpia},
    \etal
}
\altaffiltext{1}{To whom correspondence should be addressed:
                        \url{danfm@nyu.edu}}
\altaffiltext{\nyu}{Center for Cosmology and Particle Physics,
                        Department of Physics, New York University,
                        4 Washington Place, New York, NY, 10003, USA}
\altaffiltext{\mpia}{Max-Planck-Institut f\"ur Astronomie,
                        K\"onigstuhl 17, D-69117 Heidelberg, Germany}

\begin{abstract} % should be aims, methods, results
As precise as \kepler\ is,
it provides only probabilistic information
about transiting companions in the lightcurves of stars.
The official \kepler\ best-fit values for the parameters
(periods, radii ratios, durations, impact parameters)
contain some known biases,
are not associated with full uncertainty propagation,
and cannot be used as the basis of any probabilistic population analysis.

Here we provide a catalog of probabilistic constraints on the physical parameters
for every planet candidate (KOI) from the \kepler\ transiting
planet search.
Using a physically motivated generative model with flat ``interim'' priors, we
draw samples from the posterior probability distribution conditioned on all
the available light curves from \kepler\ Quarters 0 through 16.
We release both the catalog values and full $16384$-element posterior samplings for each
system.
We compare our results to the existing KOI Catalog and find xxx.
Using these samplings, we demonstrate a computationally tractable hierarchical
inference technique for simultaneously modeling the exoplanet radius
distribution and the selection function of the \kepler\ survey and pipeline.
When applied to a sample of cool stars with well constrained properties, we
find yyy.

Access to the catalog is available on-line at \url{http://data.kplr.co} and
the code is available---under the terms of the MIT license---at
\url{https://github.com/dfm/kois}.

\end{abstract}

\keywords{% CUT THESE DOWN TO SIX
  astronomical~databases:~miscellaneous
  ---
  catalogs
  ---
  eclipses
  ---
  methods:~statistical
  ---
  planetary~systems
  ---
  planets~and~satellites:~detection
  ---
  planets~and~satellites:~fundamental~parameters
  ---
  stars:~statistics
  ---
  surveys
}

\section{Introduction}

So probabilistically righteous.

TODO: Talk about the philosophy of interim priors here.

TODO: Justify the use of non-physically motivated parameters and relate this
to the idea of sufficient statistics.

\section{Data}

Kepler data here.

Cut out bits around the transits, masking, etc.

Pre-conditioning by doing the local linear fits.

What gets passed forward:  Flattened light curve bits plus ivars?

\section{Model}

A model, from our perspective, is a parameterized probability for the data,
the likelihood function, and a prior probability distribution over parameters.
We will also divide the likelihood further into a \emph{deterministic physical
generative model} and a \emph{stochastic noise model}.
The form for each of these model components can be freely chosen and in the
next sections, we'll justify our specific choices.
In short, we assume independent Gaussian noise, a Mandel \& Agol (CITE) like
limb darkened light curve, and simple interim priors.

\paragraph{Noise model}
The noise in the light curves generated by the Kepler pipeline result from a
large collection of effects.
There are both astrophysical sources (stellar variability, \etc) instrumental
effects (photon noise, temperature and pointing variations, \etc).
The exact effect of each of these effects is complicated by the aperture
photometry procedure and all of the preconditioning steps discussed above.
Despite these complications, we'll assume that the error bars produced by the
Kepler pipeline are good estimates and that they are distributed independently
and normally.
This is a reasonable assumption because TODO.

\paragraph{Generative model}
For this project, we assume a quadratic limb darkened light curve model
following Mandel \& Agol (CITE) and integrate over exposure time (Kipping
CITE, Appendix XXXX).
Since we don't want to be dependent on specific stellar parameters and since
we would like to make it easy to apply any new stellar catalogs, we don't use
physical orbits.
Instead, we parameterize the orbit using only ``observable'' parameters.
In particular, we use, for the star, the mean stellar flux \fstar, and the
parameters \qone\ and \qtwo\ of a quadratic limb darkening profile
(parameterized following Kipping CITE).
For each companion, we parameterize the orbit using period \period, epoch
\epoch, transit duration \duration, radius ratio \ror, and impact parameter
\impact.

\paragraph{Interim priors}
For simplicity---and that is the main goal when using interim priors---we
chose to model the prior probabilities for each parameter as uniform
distributions in the linear parameter except the radius ratio.
For this distribution, we use
\begin{eqnarray}
p(\ror) &\propto& \left \{ \begin{array}{ll}
1/\ror^2 & \ror_\mathrm{min} \le \ror \le 1 \\
0 & \mathrm{otherwise} \\
\end{array}\right.
\end{eqnarray}
where we conservatively choose $\ror_\mathrm{min} = 10^{-4}$.

\begin{deluxetable}{lclcc}
\tabletypesize{\footnotesize}
\tablecolumns{3}
\tablewidth{0pt}

\tablecaption{%
The parameters of the generative model and their associated interim priors.
\tablabel{parameters}}

\tablehead{%
\colhead{Symbol} &
\colhead{Unit} &
\colhead{Description} &
\colhead{Range} &
\colhead{Prior}
}

\startdata

\fstar & --- & mean stellar flux & $[0, 2]$ & Uniform \\
\qone & --- & first limb darkening coefficient & $[0, 1]$ & Uniform \\
\vspace{0.5cm}
\qtwo & --- & second limb darkening coefficient & $[0, 1]$ & Uniform \\

\period & days & orbital period & $[0, 1000]$\tablenotemark{a} & Uniform \\
\epoch & days & center time of reference transit & $[0, 1000]$\tablenotemark{a}
       & Uniform \\
\duration & days & transit duration\tablenotemark{b} & $[0, 10]$ & Uniform \\
\ror & --- & radius ratio & $[10^{-4}, 1]$ & $1/\ror^2$ \\
\impact & --- & impact parameter & $[0, 2]$ & Uniform \\

\enddata

\tablenotetext{a}{%
In practice, we use more constraining priors on \period\ and \epoch\
because the initial values are very close to correct in most cases but this
doesn't affect the results.
}
\tablenotetext{b}{%
The transit duration is defined as the time between initial contact and
final contact.
}
\end{deluxetable}


For the probability of the data, we do [XXX chi-squared].

We choose to sample in parameters that are directly informed by the light
curve instead of the full set of physical parameters in the generative model.
The motivation for this choice is to attempt to find parameters that are close
to sufficient statistics for the light curves while being independent of the
stellar parameters.
To this end, we chose to parameterize each planet candidate with period
\period, epoch \epoch, radius \ror\ (measured in stellar radii), impact
parameter \impact, and transit duration \duration.
Our specific definition of transit duration deserves some discussion.
We define the transit duration as the time that it would take for the
center of the exoplanet to transit the center of the star (at $b=0$) instead
of choosing one of the standard definitions---for example $T_\mathrm{tot}$
from \citet{winn}.
We make the further simplifying assumption that the planet transits in a
straight line across the face of the star (WORDING).
In this limit, the conversion between \duration\ and the standard
$T_\mathrm{tot}$ used in the KOI catalog is
\begin{eqnarray}
T_\mathrm{tot} \approx \duration \, \sqrt{(1+\ror)^2 - \impact^2}
\end{eqnarray}

\section{Comments}

\begin{itemize}
\item{Limb darkening}
\item{Integration over exposure time}
\end{itemize}

\section{The empirical radius distribution}

The problem of determining the empirical distribution of exoplanet radii
without taking selection effects into account (see Foreman-Mackey \etal\ in
prep), can be written as a hierarchical inference problem.
We have samples from the posterior probability distribution
\begin{eqnarray}
z^{(j)} &\sim& p(z\,|\,x,\,\alpha_0)
\end{eqnarray}
where $z$ is the radius ratio, $x$ is the light curve, and $\alpha_0$ are the
parameters of the interim prior $p(z\,|\,\alpha_0)$.
Instead, we would like to compute and/or sample the parameters describing the
distribution of physical radii $p(r\,|\,\alpha)$.
To do this, we must compute the \emph{marginalized likelihood}
\begin{eqnarray}\eqlabel{marg-like}
p(x,\,\hat{R}\,|\,\alpha) &=&
    \int \dd r\dd R\dd z\,\prod_n p(R_n)\,p(\hat{R}_n\,|\,R_n)
        \prod_k p(r_{nk}\,|\,\alpha)\,p(z_{nk}\,|\,R_n,\,r_{nk})\,
                p(x\,|\,z_{nk}) \quad.
\end{eqnarray}
We can rewrite the inner term---using the fact that
$p(z_{nk}\,|\,R_n,\,r_{nk})\equiv\delta(r_{nk}-R_n\cdot z_{nk})$---as
\begin{eqnarray}
p_{nk}(x\,|\,R_n,\,\alpha)&=&
\int \dd r_{nk}\dd z_{nk}\,p(r_{nk}\,|\,\alpha)\,p(z_{nk}\,|\,R_n,\,r_{nk})\,
        p(x\,|\,z_{nk}) \nonumber\\
&=&
\int \dd z_{nk}\,p(z_{nk} \cdot R_n\,|\,\alpha)\,p(x\,|\,z_{nk}) \nonumber\\
&=&
\int \dd z_{nk}\,p(z_{nk}\cdot R_n\,|\,\alpha)\,p(x\,|\,z_{nk})\,
\frac{p(z_{nk}\,|\,x,\,\alpha_0)}{p(z_{nk}\,|\,x,\,\alpha_0)} \nonumber\\
&\propto&
\int \dd z_{nk}\,\frac{p(z_{nk}\cdot R_n\,|\,\alpha)}{p(z_{nk}\,|\,\alpha_0)}\,
p(z_{nk}\,|\,x,\,\alpha_0) \nonumber\\
&\approx&
\frac{1}{J_{nk}}
\sum_j \frac{p(z_{nk}^{(j)}\cdot R_n\,|\,\alpha)}{p(z_{nk}^{(j)}\,|\,\alpha_0)}
\end{eqnarray}

The integral over stellar radius can also be approximated using posterior
samples
\begin{eqnarray}
R^{(l)} &\sim& p(R\,|\,\hat{R}) \quad.
\end{eqnarray}
Here, we're not trying to simultaneously infer the distribution of stellar
parameters.
This simplifies matters and the full integral in \eq{marg-like} can be
approximated as
\begin{eqnarray}
\frac{p(x,\,\hat{R}\,|\,\alpha)}{p(x,\,\hat{R}\,|\,\alpha_0)}
&\approx& \prod_n \frac{1}{L_n} \sum_{l=1}^{L_n} \prod_k
\frac{1}{J_{nk}} \sum_{j=1}^{J_{nk}}
\frac{p(z_{nk}^{(j)}\cdot R_n\,|\,\alpha)}{p(z_{nk}^{(j)}\,|\,\alpha_0)}
\quad.
\end{eqnarray}
This ratio can now be computed efficiently for different values $\alpha$ and
optimized or you can set priors $p(\alpha)$ and sample.

\acknowledgments
It is a pleasure to thank
    Tom Barclay (NASA Ames),
    Tim Morton (Princeton),
    \ldots
for helpful contributions to the ideas and code presented here.
This project was partially supported by the NSF (grant AST-0908357), and NASA
(grant NNX08AJ48G).

\newcommand{\arxiv}[1]{\href{http://arxiv.org/abs/#1}{arXiv:#1}}
\begin{thebibliography}{}\raggedright

\bibitem[Tremaine \& Dong(2012)]{tremaine}
Tremaine, S., \& Dong, S.\ 2012, \aj, 143, 94
\arxiv{1106.5403}

\bibitem[Winn(2010)]{winn}
Winn, J.~N.\ 2010, \arxiv{1001.2010}

\end{thebibliography}

\end{document}
