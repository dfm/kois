\documentclass[12pt,preprint]{aastex}

% has to be before amssymb it seems
\usepackage{color,hyperref}
\definecolor{linkcolor}{rgb}{0,0,0.5}
\hypersetup{colorlinks=true,linkcolor=linkcolor,citecolor=linkcolor,
            filecolor=linkcolor,urlcolor=linkcolor}

\usepackage{url}
\usepackage{algorithmic,algorithm}
\usepackage{multirow}

\usepackage{listings}
\definecolor{lbcolor}{rgb}{0.9,0.9,0.9}
\lstset{language=Python,
        basicstyle=\footnotesize\ttfamily,
        showspaces=false,
        showstringspaces=false,
        tabsize=2,
        breaklines=false,
        breakatwhitespace=true,
        identifierstyle=\ttfamily,
        keywordstyle=\bfseries\color[rgb]{0.133,0.545,0.133},
        commentstyle=\color[rgb]{0.133,0.545,0.133},
        stringstyle=\color[rgb]{0.627,0.126,0.941},
    }

\usepackage{amssymb,amsmath}

\newcommand{\project}[1]{{\sffamily #1}}
\newcommand{\Python}{\project{Python}}
\newcommand{\numpy}{\project{numpy}}
\newcommand{\bart}{\project{Bart}}
\newcommand{\emcee}{\project{emcee}}
\newcommand{\kepler}{\project{Kepler}}
\newcommand{\kplr}{\project{kplr}}
\newcommand{\license}{MIT License}

\newcommand{\paper}{\emph{Article}}

\newcommand{\foreign}[1]{\emph{#1}}
\newcommand{\etal}{\foreign{et\,al.}}
\newcommand{\etc}{\foreign{etc.}}

\newcommand{\Fig}[1]{Figure~\ref{fig:#1}}
\newcommand{\fig}[1]{\Fig{#1}}
\newcommand{\figlabel}[1]{\label{fig:#1}}
\newcommand{\Tab}[1]{Table~\ref{tab:#1}}
\newcommand{\tab}[1]{\Tab{#1}}
\newcommand{\tablabel}[1]{\label{tab:#1}}
\newcommand{\Eq}[1]{Equation~(\ref{eq:#1})}
\newcommand{\eq}[1]{\Eq{#1}}
\newcommand{\eqlabel}[1]{\label{eq:#1}}
\newcommand{\Sect}[1]{Section~\ref{sect:#1}}
\newcommand{\sect}[1]{\Sect{#1}}
\newcommand{\App}[1]{Appendix~\ref{sect:#1}}
\newcommand{\app}[1]{\App{#1}}
\newcommand{\sectlabel}[1]{\label{sect:#1}}
\newcommand{\Algo}[1]{Algorithm~\ref{algo:#1}}
\newcommand{\algo}[1]{\Algo{#1}}
\newcommand{\algolabel}[1]{\label{algo:#1}}

% math symbols
\newcommand{\dd}{\ensuremath{\,\mathrm{d}}}
\newcommand{\bvec}[1]{\ensuremath{\boldsymbol{#1}}}
\newcommand{\unit}[1]{\ensuremath{\mathrm{#1}}}
\newcommand{\normal}[1]{\ensuremath{\mathcal{N}(#1)}}

\newcommand{\obs}[1]{\ensuremath{\overline{#1}}}

% document symbols
\newcommand{\population}{\ensuremath{\alpha}}
\newcommand{\planet}{\ensuremath{w}}
\newcommand{\planetobs}{\ensuremath{\obs{w}}}
\newcommand{\stellar}{\ensuremath{s}}
\newcommand{\stellarobs}{\ensuremath{\obs{s}}}
\newcommand{\isobs}{\ensuremath{q}}
\newcommand{\selection}{\ensuremath{\Delta}}

% parameters
\newcommand{\period}{\ensuremath{P}}
\newcommand{\relincl}{{\ensuremath{\delta i}}}
\newcommand{\radius}{\ensuremath{r}}
\newcommand{\periodobs}{\ensuremath{\obs{P}}}
\newcommand{\rorobs}{{\ensuremath{\obs{r/\sradius}}}}
\newcommand{\impactobs}{\ensuremath{\obs{b}}}

\newcommand{\smass}{\ensuremath{M}}
\newcommand{\sradius}{\ensuremath{R}}
\newcommand{\snoise}{\ensuremath{\sigma}}
\newcommand{\incl}{\ensuremath{i}}
\newcommand{\snoiseobs}{\ensuremath{\obs{\sigma}}}
\newcommand{\sloggobs}{{\ensuremath{\obs{\log g}}}}


\begin{document}

\title{%
The \kplr\ Catalog:
Systematic probabilistic parameter estimation for every \kepler\ planet candidate
}

\newcommand{\nyu}{2}
\newcommand{\mpia}{3}
\author{%
    Daniel~Foreman-Mackey\altaffilmark{1,\nyu},
    David~W.~Hogg\altaffilmark{\nyu,\mpia},
    \etal
}
\altaffiltext{1}{To whom correspondence should be addressed:
                        \url{danfm@nyu.edu}}
\altaffiltext{\nyu}{Center for Cosmology and Particle Physics,
                        Department of Physics, New York University,
                        4 Washington Place, New York, NY, 10003, USA}
\altaffiltext{\mpia}{Max-Planck-Institut f\"ur Astronomie,
                        K\"onigstuhl 17, D-69117 Heidelberg, Germany}

\begin{abstract} % should be aims, methods, results
As precise as \kepler\ is,
it provides only probabilistic information
about transiting companions in the lightcurves of stars.
The official \kepler\ best-fit values for the parameters
(periods, radii ratios, durations, impact parameters)
contain some known biases,
are not associated with full uncertainty propagation,
and cannot be used as the basis of any probabilistic population analysis.

Here we provide a catalog of probabilistic constraints on the physical parameters
for every planet candidate (KOI) from the \kepler\ transiting
planet search.
Using a physically motivated generative model with flat ``interim'' priors, we
draw samples from the posterior probability distribution conditioned on all
the available light curves from \kepler\ Quarters 0 through 16.
We release both the catalog values and full $16384$-element posterior samplings for each
system.
We compare our results to the existing KOI Catalog and find xxx.
Using these samplings, we demonstrate a computationally tractable hierarchical
inference technique for simultaneously modeling the exoplanet radius
distribution and the selection function of the \kepler\ survey and pipeline.
When applied to a sample of cool stars with well constrained properties, we
find yyy.

Access to the catalog is available on-line at \url{http://data.kplr.co} and
the code is available---under the terms of the MIT license---at
\url{https://github.com/dfm/kois}.

\end{abstract}

\keywords{% CUT THESE DOWN TO SIX
  astronomical~databases:~miscellaneous
  ---
  catalogs
  ---
  eclipses
  ---
  methods:~statistical
  ---
  planetary~systems
  ---
  planets~and~satellites:~detection
  ---
  planets~and~satellites:~fundamental~parameters
  ---
  stars:~statistics
  ---
  surveys
}

\section{Introduction}

So probabilistically righteous.

\section{Parameters}

We choose to sample in parameters that are directly informed by the light
curve instead of the full set of physical parameters in the generative model.
The motivation for this choice is to attempt to find parameters that are close
to sufficient statistics for the light curves while being independent of the
stellar parameters.
To this end, we chose to parameterize each planet candidate with period
\period, epoch \epoch, radius \ror\ (measured in stellar radii), impact
parameter \impact, and transit duration \duration.
Our specific definition of transit duration deserves some discussion.
We define the transit duration as the time that it would take for the
center of the exoplanet to transit the center of the star (at $b=0$) instead
of choosing one of the standard definitions---for example $T_\mathrm{tot}$
from \citet{winn}.
We make the further simplifying assumption that the planet transits in a
straight line across the face of the star (WORDING).
In this limit, the conversion between \duration\ and the standard
$T_\mathrm{tot}$ used in the KOI catalog is
\begin{eqnarray}
T_\mathrm{tot} \approx \duration \, \sqrt{(1+\ror)^2 - \impact^2}
\end{eqnarray}

\section{Comments}

\begin{itemize}
\item{Limb darkening}
\item{Integration over exposure time}
\end{itemize}

\acknowledgments
It is a pleasure to thank
    Tom Barclay (NASA Ames),
    Tim Morton (Princeton),
    \ldots
for helpful contributions to the ideas and code presented here.
This project was partially supported by the NSF (grant AST-0908357), and NASA
(grant NNX08AJ48G).

\newcommand{\arxiv}[1]{\href{http://arxiv.org/abs/#1}{arXiv:#1}}
\begin{thebibliography}{}\raggedright

\bibitem[Tremaine \& Dong(2012)]{tremaine}
Tremaine, S., \& Dong, S.\ 2012, \aj, 143, 94
\arxiv{1106.5403}

\bibitem[Winn(2010)]{winn}
Winn, J.~N.\ 2010, \arxiv{1001.2010}

\end{thebibliography}

\end{document}
