\documentclass[12pt,preprint]{aastex}

% has to be before amssymb it seems
\usepackage{color,hyperref}
\definecolor{linkcolor}{rgb}{0,0,0.5}
\hypersetup{colorlinks=true,linkcolor=linkcolor,citecolor=linkcolor,
            filecolor=linkcolor,urlcolor=linkcolor}

\usepackage{url}
\usepackage{algorithmic,algorithm}
\usepackage{multirow}

\usepackage{listings}
\definecolor{lbcolor}{rgb}{0.9,0.9,0.9}
\lstset{language=Python,
        basicstyle=\footnotesize\ttfamily,
        showspaces=false,
        showstringspaces=false,
        tabsize=2,
        breaklines=false,
        breakatwhitespace=true,
        identifierstyle=\ttfamily,
        keywordstyle=\bfseries\color[rgb]{0.133,0.545,0.133},
        commentstyle=\color[rgb]{0.133,0.545,0.133},
        stringstyle=\color[rgb]{0.627,0.126,0.941},
    }

\input{vars}

\begin{document}

\title{%
The \kplr\ Catalog:
Systematic probabilistic parameter estimation for all \kepler\ planet
candidates and inference of the radius distribution
}

\newcommand{\nyu}{2}
\newcommand{\mpia}{3}
\author{%
    Daniel~Foreman-Mackey\altaffilmark{1,\nyu},
    David~W.~Hogg\altaffilmark{\nyu,\mpia},
    \etal
}
\altaffiltext{1}{To whom correspondence should be addressed:
                        \url{danfm@nyu.edu}}
\altaffiltext{\nyu}{Center for Cosmology and Particle Physics,
                        Department of Physics, New York University,
                        4 Washington Place, New York, NY, 10003, USA}
\altaffiltext{\mpia}{Max-Planck-Institut f\"ur Astronomie,
                        K\"onigstuhl 17, D-69117 Heidelberg, Germany}

\begin{abstract}

We provide a catalog of probabilistic constraints on the physical parameters
of each of the planet candidates found using data from the \kepler\ transiting
planet search.
Using a physically motivated generative model with flat ``default'' priors, we
draw samples from the posterior probability distribution conditioned on all
the available light curves from \kepler\ Quarters 0 through 16.
We release both the catalog values and full posterior samplings for each
system.
We compare our results to the existing KOI catalog and find xxx.
Using these samplings, we demonstrate a computationally tractable hierarchical
inference technique for simultaneously modeling the exoplanet radius
distribution and the selection function of the \kepler\ survey and pipeline.
When applied to a sample of cool stars with well constrained properties, we
find yyy.

Access to the catalog is available on-line at \url{http://data.kplr.co} and
the code is available---under the terms of the MIT license---at
\url{https://github.com/dfm/kois}.

\end{abstract}

\keywords{% CUT THESE DOWN TO SIX
  astronomical~databases:~miscellaneous
  ---
  catalogs
  ---
  eclipses
  ---
  methods:~statistical
  ---
  planetary~systems
  ---
  planets~and~satellites:~detection
  ---
  planets~and~satellites:~fundamental~parameters
  ---
  stars:~statistics
  ---
  surveys
}

\section{Introduction}

So probabilistically righteous.

\acknowledgments
It is a pleasure to thank
    Tom Barclay (NASA Ames),
    Tim Morton (Princeton),
    \ldots
for helpful contributions to the ideas and code presented here.
This project was partially supported by the NSF (grant AST-0908357), and NASA
(grant NNX08AJ48G).

\newcommand{\arxiv}[1]{\href{http://arxiv.org/abs/#1}{arXiv:#1}}
\begin{thebibliography}{}\raggedright

\bibitem[Tremaine \& Dong(2012)]{tremaine}
Tremaine, S., \& Dong, S.\ 2012, \aj, 143, 94
\arxiv{1106.5403}

\end{thebibliography}

\end{document}
